\section{Discussion}

% Write a critical discussion about your simulations. What are the major shortcomings of your approach? Is your model still useful? What can be done to make it better? (10 points)

In this study, a two-dimensional ice flow model for the GrIS was implemented from scratch. After testing and benchmarking, the model was used to analyse the GrIS's response to varying climate conditions, to separate the effects of temperature and precipitation perturbations, and to project the future of the GrIS under global warming scenarios. While the low-complexity model is time-efficient, the simplicity causes problems when initializing the model with an idealized SMB derived from present-day climate. Those simulations neither left the GrIS unchanged nor decreased the ice volume, but instead showed an increase of the ice sheet to almost twice its size. Whether the scaling processes to calculate the SMB are over-simplified, in need of improved calibration, and/or the model lacks important internal mechanisms (e.g., bedrock sinking, basal hydrology) requires further analysis. 

However, if relying on differences to a reference simulation, rather than absolute values, the model produces accurate results that also align with published estimates. It is able to reproduce the combined effect of increased temperature and precipitation resulting in an accumulation of the ice sheet in central Greenland. Generally speaking, when comparing equilibrium states of differing synthetic climate conditions, the model produces reasonable results. Additionally, if applying a realistic global warming scenario for a limited amount of simulated years, the large-scale changes are consistent with published estimates. Of course, if research requires high spatial resolution or is focussed on only a small region of Greenland, this model is not suitable, rather, it can act as a starting point of an analysis, to eventually move to more complex (and time-consuming) GrIS models.

A possible future analysis can study the impact of changes to the background climate signal, similar to what has been started with the NorESM forcing, but as the model can easily simulate thousands of years, on longer time-scales. If incorporating the Milankovitch cycles into the climate forcing, can the model produce corresponding glacial and interglacial frequencies?    

%Distance to coastline is scaled by number of grid points and has been adjusted on high resolution grid!
