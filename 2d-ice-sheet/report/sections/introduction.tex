\section{Introduction}\label{sec:intro}

% TODO:
% - Abstract

% Write an introduction on the state of the Greenland ice sheet today and what is known about its future stability. What is the state of the art and what can be done to improve future projections of sea level rise contribution? Use the papers that were presented in class. (10 points)

The Greenland ice sheet (GrIS) is the second-largest body of ice in the world and covers roughly \SI{80}{\percent} of Greenland \parencite{raikar2024}. Its average thickness is around \SI{1.5}{\km}, however, the ice can reach maximum depths of over \SI{3}{\km}. 

The GrIS shrinks as climate warms, a process that can already be observed at the present day \parencite[Ch.\ 14]{cuffey2010}. While the net balance of the entire ice sheet was only slightly negative from 1960 to 1990, increased melting has occurred in subsequent years. Ice loss has generally accelerated in recent years, yet it is also superimposed by large year-to-year fluctuations. The rate of ice loss corresponded to a global sea level rise of \SIrange{0.5}{0.6}{\mm\per\year} by 2006.

The ice sheet can display non-linear responses to climate forcings due to self-perpetuating positive feedbacks \parencite{aschwanden2019} such as:
\begin{itemize}
	\item Ocean warming triggers the acceleration of outlet glaciers by disintegration of their floating ice tongues (outlet glacier–acceleration feedback).
	\item Surface melt lowers the elevation of the ice sheet and thereby exposes it to higher temperatures (surface mass balance–elevation feedback).
\end{itemize}   

Hence, uncertainties in predicting future sea level rise are predominately caused by uncertainties in the climate scenarios and surface processes. Model simulations suggest that the reduction of the GrIS could contribute \SIrange{5}{33}{\cm} to sea level by 2100, and it is projected that Greenland will very likely be ice-free within a millenium under the RCP8.5 scenario. Better quantification of these feedbacks might reveal at which point the reduction of the GrIS becomes non-reversible.